\capitulo{2}{Objetivos del proyecto}

% Este apartado explica de forma precisa y concisa cuales son los objetivos que se persiguen con la realización del proyecto. Se puede distinguir entre los objetivos marcados por los requisitos del software a construir y los objetivos de carácter técnico que plantea a la hora de llevar a la práctica el proyecto.


A continuación se van a detallar los objetivos perseguidos con la realización del proyecto así como los de la aplicación Web sobra la que se trabaja,



\section{Objetivos de la aplicación y funcionalidad previa}
   
A continuación se enumeran los objetivos iniciales de la aplicación ya desarrollada y cómo se han desarrollado: \cite{TFGPrevio}, un software para calcular métricas de control \footnote{También llamadas métricas de proceso o métricas de evolución} sobre distintos repositorios.
\begin{itemize}
	\tightlist
	\item Se obtienen medidas de métricas de evolución de uno o varios proyectos alojados en repositorios de GitLab.
	\item Las métricas que se calculan de un repositorio  son algunas de las especificadas en la tesis titulada ``\textit{sPACE: Software Project Assessment in the Course of Evolution}'' \cite{ratzinger_space:_2007} y 
	adaptadas a los repositorios software:
	\begin{itemize}
		\tightlist
		\item Número total de incidencias (\textit{issues})
		\item Cambios (\textit{commits}) por incidencia
		\item Porcentaje de incidencias cerrados
		\item Media de días en cerrar una incidencia
		\item Media de días entre cambios
		\item Días entre primer y último cambio
		\item Rango de actividad de cambios por mes
		\item Porcentaje de pico de cambios
	\end{itemize}
	\item Se permite comparar con otros proyectos de la misma naturaleza. Para ello se establecen unos valores umbrales por cada métrica basados en el cálculo de los cuartiles Q1 y Q3. Además, estos valores se calculan dinámicamente y se almacenan en perfiles de configuración de métricas.
	\item Se permite la posibilidad de almacenar de manera persistente estos perfiles de configuiración de métricas para permitir comparaciones futuras.
	\item También se permite almacenar de forma persistente las métricas obtenidas de los repositorios para su posterior consulta o tratamiento. Esto permite comparar nuevos proyectos con proyectos de los que ya se han calculado sus métricas.
	 Esta funcionalidad se basa en la exportación e importación de los valores de métricas obtenidos por la aplicación en diferentes análisis usando archivos CSV.
\end{itemize}

\newpage

\section{Objetivos generales}
El objetivo principal del presente TFG no se puede entender sin tener en cuenta los objetivos previamente enumerados ya que es realizar mejoras y extender la funcionalidad que tiene el software \textit{\textbf{Evolution Metrics Gauge}}.
En esta nueva iteración se pretende:

\begin{itemize}
	\tightlist
	\item Permitir leer repositorios de GitHub además de GitLab.
	\item Introducir nuevas métricas a las ya evaluadas. TODO -> definir cuáles
	\item Realizar pruebas con repositorios de GitLab y GitHub simultáneamente.
	\item Realizar diferentes mejoras en las interfaces de la aplicación para mejorar la experiencia de usuario.
	\item Corrección de errores.
	\item Mejorar el paquete de tests de la aplicación para mejorar su cobertura.
\end{itemize}

\newpage


\section{Objetivos técnicos proyecto existente}
Este apartado recoge los requisitos técnicos del proyecto existente  \cite{TFGPrevio}:
\begin{itemize}
	\tightlist
	\item Diseño de la aplicación de manera que se puedan extender con nuevas métricas con el menor coste de mantenimiento posible. Para ello, se aplica un diseño basado en frameworks y en patrones de diseño \cite{gamma_patrones_2002}.
	\item El diseño de la aplicación facilita la extensión a otras plataformas de desarrollo colaborativo como GitHub o Bitbucket.
	\item Aplicación del \textit{frameworks `modelo-vista-controlador'} para separar la lógica de la aplicación y la interfaz de usuario.
	\item Creación una batería de pruebas automáticas con cobertura por encima del 90\% en los subsistemas de lógica de la aplicación.
	\item Utilización una plataforma de desarrollo colaborativo que incluya un sistema de control de versiones, un sistema de seguimiento de incidencias y que permita una comunicación fluida entre el tutor y el alumno.
	\item Utilización un sistema de integración y despliegue continuo.
	\item Correcta gestión de errores definiendo excepciones de biblioteca y registrando eventos de error e información en ficheros de \textit{log}. 
	\item Aplicar nuevas estructuras  del lenguaje Java para el desarrollo, como son expresiones lambda. 
	\item Utilización de sistemas que aseguren la calidad continua del código que permitan evaluar la deuda técnica del proyecto.
	\item Pruebas la aplicación con ejemplos reales y utilizando técnicas avanzadas, como entrada de datos de test en ficheros con formato tabulado tipo CSV (\textit{comma separated values}). 	
\end{itemize}

\newpage

\section{Objetivos técnicos}
Además de mantener todos los objetivos técnicos previos fijados para el desarrollo de la aplicación en la primera iteración, se busca profundizar en ellos y mejorar el desarrollo actual. Para ello se se proponen los siguientes objetivos:
\begin{itemize}
	\tightlist
	\item Mejora de la cobertura de pruebas automáticas del proyecto. 
	\item Mejora del tratamiento de errores para evitar bloqueos de la aplicación.
	\item Mejora del almacenamiento y visualización de errores para facilitar su corrección de cara a posteriores iteraciones.
	\item Actualización y puesta a punto del sistema de integración y despliegue continuo.
	\item Pruebas la aplicación con ejemplos reales y utilizando técnicas avanzadas, como entrada de datos de test en ficheros con formato tabulado tipo CSV (\textit{comma separated values}) también para métricas obtenidas de repositorios de GitHub. 	
\end{itemize}



