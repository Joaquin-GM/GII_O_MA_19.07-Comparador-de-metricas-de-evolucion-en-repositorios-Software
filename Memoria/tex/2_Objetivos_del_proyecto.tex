\capitulo{2}{Objetivos del proyecto}

% Este apartado explica de forma precisa y concisa cuales son los objetivos que se persiguen con la realización del proyecto. Se puede distinguir entre los objetivos marcados por los requisitos del software a construir y los objetivos de carácter técnico que plantea a la hora de llevar a la práctica el proyecto.

Este TFG es una segunda iteración del software desarrollado \textit{\textbf{Evolution Metrics Gauge}} y disponible en: 
\\ \url{https://gitlab.com/mlb0029/comparador-de-metricas-de-evolucion-en-repositorios-software}
  
Para facilitar la comprensión se divide en esta sección con los objetivos definidos en cada iteración. Además se ha diferenciado entre objetivos generales y técnicos.

\section{Objetivos Evolution Metrics Gauge iteración 1}
A continuación se enumeran los objetivos iniciales de la aplicación ya desarrollada y cómo se han desarrollado: \cite{TFGPrevio}
\begin{itemize}
	\tightlist
	\item Se obtienen medidas de métricas de evolución de uno o varios proyectos alojados en repositorios de GitLab.
	\item Las métricas que se calculan de un repositorio  son algunas de las especificadas en la tesis titulada ``\textit{sPACE: Software Project Assessment in the Course of Evolution}'' \cite{ratzinger_space:_2007} y 
	adaptadas a los repositorios software:
	\begin{itemize}
		\tightlist
		\item Número total de incidencias (\textit{issues})
		\item Cambios (\textit{commits}) por incidencia
		\item Porcentaje de incidencias cerrados
		\item Media de días en cerrar una incidencia
		\item Media de días entre cambios
		\item Días entre primer y último cambio
		\item Rango de actividad de cambios por mes
		\item Porcentaje de pico de cambios
	\end{itemize}
	\item Se permite comparar con otros proyectos de la misma naturaleza. Para ello se establecen unos valores umbrales por cada métrica basados en el cálculo de los cuartiles Q1 y Q3. Además, estos valores se calculan dinámicamente y se almacenan en perfiles de configuración de métricas.
	\item Se permite la posibilidad de almacenar de manera persistente estos perfiles de configuración de métricas para permitir comparaciones futuras (TODO -> comprobar)
	\item También se permite almacenar de forma persistente las métricas obtenidas de los repositorios para su posterior consulta o tratamiento. Esto permite comparar nuevos proyectos con proyectos de los que ya se han calculado sus métricas.
	 Esta funcionalidad se basa en la exportación e importación de los valores de métricas obtenidos por la aplicación en diferentes análisis usando archivos CSV.
\end{itemize}


\section{Objetivos Evolution Metrics Gauge iteración 2}
   
El objetivo principal del presente TFG es realizar mejoras y extender la funcionalidad que tiene el software \textit{\textbf{Evolution Metrics Gauge}}, un software para calcular métricas de control \footnote{También llamadas métricas de proceso o métricas de evolución} sobre distintos repositorios.
En esta nueva iteración se pretende:

\begin{itemize}
	\tightlist
	\item Calcular las 8 métricas de evolución definidas sobre repositorios GitHub además de GitLab .
	\item Implementar nuevas métricas a las ya evaluadas. TODO -> definir cuáles
	\item Realizar pruebas con repositorios de GitLab y GitHub simultáneamente.
	\item Diseñar una interfaz gráfica que permita la interacción simultánea con repositorios de GitLab y GitHub.
	\item Comprender y aplicar el flujo de trabajo de integración continua del proyecto actual.
	\item Definir un conjunto de pruebas que ayuden a detectar errores de la versión actual.
\end{itemize}

\newpage


\section{Objetivos técnicos}
Este apartado recoge los requisitos técnicos del proyecto existente  \cite{TFGPrevio}:
\begin{itemize}
	\tightlist
	\item Diseño de la aplicación de manera que se puedan extender con nuevas métricas con el menor coste de mantenimiento posible. Para ello, se aplica un diseño basado en frameworks y en patrones de diseño \cite{gamma_patrones_2002}.
	\item El diseño de la aplicación facilita la extensión a otras plataformas de desarrollo colaborativo como GitHub o Bitbucket.
	\item Aplicación del \textit{frameworks `modelo-vista-controlador'} para separar la lógica de la aplicación y la interfaz de usuario.
	\item Creación una batería de pruebas automáticas con cobertura por encima del 90\% en los subsistemas de lógica de la aplicación.
	\item Utilización una plataforma de desarrollo colaborativo que incluya un sistema de control de versiones, un sistema de seguimiento de incidencias y que permita una comunicación fluida entre el tutor y el alumno.
	\item Utilización un sistema de integración y despliegue continuo.
	\item Correcta gestión de errores definiendo excepciones de biblioteca y registrando eventos de error e información en ficheros de \textit{log}. 
	\item Aplicar nuevas estructuras  del lenguaje Java para el desarrollo, como son expresiones lambda. 
	\item Utilización de sistemas que aseguren la calidad continua del código que permitan evaluar la deuda técnica del proyecto.
	\item Pruebas la aplicación con ejemplos reales y utilizando técnicas avanzadas, como entrada de datos de test en ficheros con formato tabulado tipo CSV (\textit{comma separated values}).
	\item Comprender y aplicar el flujo de trabajo de integración continua del proyecto actual.
	\item Implementar un sistema de registro de errores persistente para gestionar su posible resolución.
	\item Pruebas la aplicación con ejemplos reales y utilizando técnicas avanzadas, como entrada de datos de test en ficheros con formato tabulado tipo CSV (\textit{comma separated values}) también para métricas obtenidas de repositorios de GitHub. 		
\end{itemize}


%\section{Objetivos técnicos Evolution Metrics Gauge iteración 1}
%\section{Objetivos técnicos Evolution Metrics Gauge iteración 2}
%Además de mantener todos los objetivos técnicos previos fijados para el desarrollo de la aplicación en la primera iteración, se busca profundizar en ellos y mejorar el desarrollo actual. Para ello se se proponen los siguientes objetivos:
%\begin{itemize}
%	\tightlist
%	\item Comprender y aplicar el flujo de trabajo de integración continua del proyecto actual.
%	\item Implementar un sistema de registro de errores persistente para gestionar su posible resolución.
%	\item Pruebas la aplicación con ejemplos reales y utilizando técnicas avanzadas, como entrada de datos de test en ficheros con formato tabulado tipo CSV (\textit{comma separated values}) también para métricas obtenidas de repositorios de GitHub. 	
%\end{itemize}
